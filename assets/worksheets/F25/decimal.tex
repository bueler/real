\documentclass[11pt]{amsart}
%prepared in AMSLaTeX, under LaTeX2e
\addtolength{\oddsidemargin}{-.9in} 
\addtolength{\evensidemargin}{-.9in}
\addtolength{\topmargin}{-.9in}
\addtolength{\textwidth}{1.5in}
\addtolength{\textheight}{1.5in}

\renewcommand{\baselinestretch}{1.05}

\usepackage{verbatim} % for "comment" environment

\usepackage{palatino,enumitem}

\usepackage[final]{graphicx}

\usepackage{tikz}
\usetikzlibrary{positioning}

\usepackage{enumitem,xspace,fancyvrb}

\newtheorem*{thm}{Theorem}
\newtheorem*{defn}{Definition}
\newtheorem*{example}{Example}
\newtheorem*{problem}{Problem}
\newtheorem*{remark}{Remark}

\DefineVerbatimEnvironment{mVerb}{Verbatim}{numbersep=2mm,frame=lines,framerule=0.1mm,framesep=2mm,xleftmargin=4mm,fontsize=\footnotesize}

% macros
\usepackage{amssymb}
\newcommand{\bA}{\mathbf{A}}
\newcommand{\bB}{\mathbf{B}}
\newcommand{\bE}{\mathbf{E}}
\newcommand{\bF}{\mathbf{F}}
\newcommand{\bJ}{\mathbf{J}}

\newcommand{\bb}{\mathbf{b}}
\newcommand{\br}{\mathbf{r}}
\newcommand{\bv}{\mathbf{v}}
\newcommand{\bw}{\mathbf{w}}
\newcommand{\bx}{\mathbf{x}}

\newcommand{\hbi}{\mathbf{\hat i}}
\newcommand{\hbj}{\mathbf{\hat j}}
\newcommand{\hbk}{\mathbf{\hat k}}
\newcommand{\hbn}{\mathbf{\hat n}}
\newcommand{\hbr}{\mathbf{\hat r}}
\newcommand{\hbt}{\mathbf{\hat t}}
\newcommand{\hbx}{\mathbf{\hat x}}
\newcommand{\hby}{\mathbf{\hat y}}
\newcommand{\hbz}{\mathbf{\hat z}}
\newcommand{\hbphi}{\mathbf{\hat \phi}}
\newcommand{\hbtheta}{\mathbf{\hat \theta}}
\newcommand{\complex}{\mathbb{C}}
\newcommand{\ppr}[1]{\frac{\partial #1}{\partial r}}
\newcommand{\ppt}[1]{\frac{\partial #1}{\partial t}}
\newcommand{\ppx}[1]{\frac{\partial #1}{\partial x}}
\newcommand{\ppy}[1]{\frac{\partial #1}{\partial y}}
\newcommand{\ppz}[1]{\frac{\partial #1}{\partial z}}
\newcommand{\pptheta}[1]{\frac{\partial #1}{\partial \theta}}
\newcommand{\ppphi}[1]{\frac{\partial #1}{\partial \phi}}
\newcommand{\Div}{\ensuremath{\nabla\cdot}}
\newcommand{\Curl}{\ensuremath{\nabla\times}}
\newcommand{\curl}[3]{\ensuremath{\begin{vmatrix} \hbi & \hbj & \hbk \\ \partial_x & \partial_y & \partial_z \\ #1 & #2 & #3 \end{vmatrix}}}
\newcommand{\cross}[6]{\ensuremath{\begin{vmatrix} \hbi & \hbj & \hbk \\ #1 & #2 & #3 \\ #4 & #5 & #6 \end{vmatrix}}}
\newcommand{\eps}{\epsilon}
\newcommand{\grad}{\nabla}
\newcommand{\image}{\operatorname{im}}
\newcommand{\integers}{\mathbb{Z}}
\newcommand{\ip}[2]{\ensuremath{\left<#1,#2\right>}}
\newcommand{\lam}{\lambda}
\newcommand{\lap}{\triangle}
\newcommand{\note}[1]{[\scriptsize #1 \normalsize]}
\newcommand{\MatIN}[1]{\mtt{>> #1}}
\newcommand{\onull}{\operatorname{null}}
\newcommand{\rank}{\operatorname{rank}}
\newcommand{\range}{\operatorname{range}}
\renewcommand{\P}{\mathcal{P}}
\newcommand{\real}{\mathbb{R}}
\newcommand{\trace}{\operatorname{tr}}

\newcommand{\prob}[1]{\bigskip\noindent\textbf{#1.}\quad }
\newcommand{\exer}[2]{\prob{Exercise #2 on page #1}}
\newcommand{\exerpages}[2]{\prob{Exercise #2 on pages #1}}

\newcommand{\pts}[1]{(\emph{#1 pts}) }
\newcommand{\epart}[1]{\bigskip\noindent\textbf{(#1)}\quad }
\newcommand{\ppart}[1]{\,\textbf{(#1)}\quad }

\newcommand{\Matlab}{\textsc{Matlab}\xspace}
\newcommand{\Octave}{\textsc{Octave}\xspace}
\newcommand{\MO}{\textsc{Matlab}/\textsc{Octave}\xspace}

\newcommand{\boxtop}[1]{\frac{\boxed{\phantom{foo}}}{#1}}
\newcommand{\wboxtop}[1]{\frac{\boxed{\phantom{foo foo foo}}}{#1}}


\begin{document}
\scriptsize \noindent Math 401 Introduction to Real Analysis \hfill Bueler
\normalsize\medskip

\Large\centerline{\textbf{Worksheet: Decimal expansions of real numbers.}}
\medskip
\normalsize

\thispagestyle{empty}

\begin{quote}
The goals here are \emph{(1)} recall the precise meaning of decimal expansions, and \emph{(2)} initiate the idea of completeness.  Here we see, from decimal expansions, why all positive real numbers are ``least upper bounds'' for sets of rational numbers.
\end{quote}

\bigskip
\begin{enumerate}[leftmargin=-1mm]
\renewcommand{\labelenumi}{\textbf{\arabic{enumi}.}}
\item Consider the irrational real number $e = 2.718281828459045\dots$.  To confirm the meaning of this expansion, fill in the numerators:
	$$e = 2 + \boxtop{10} + \boxtop{100} + \boxtop{1000} + \boxtop{10^4} + \boxtop{10^5} + \dots$$

\bigskip
\item The terminating (``finite'') decimal approximations of $e$ are \emph{rational numbers}, that is, fractions of natural numbers.  Fill in numerators; there is no need to write in lowest terms:
\begin{align*}
2.70000000000\dots &= \frac{27}{10}  & 2.71820000000\dots &= \wboxtop{10^4} \\
2.71000000000\dots &= \boxtop{100}   & 2.71828000000\dots &= \wboxtop{10^5} \\
2.71800000000\dots &= \wboxtop{1000} & &\vdots
\end{align*}
\bigskip
\item Of course, the fractions in item \textbf{2} above are each \emph{less than} $e$:
\begin{align*}
2.70000000000\dots &< e = 2.718281828459045\dots \\
2.71000000000\dots &< e \\
2.71800000000\dots &< e \\
2.71820000000\dots &< e \\
&\vdots
\end{align*}
(\emph{Nothing to fill in here.  But please make sure you agree!})

\medskip
\item \textbf{The main idea.}  Write an infinite set of rational numbers for which $e$ is an \emph{upper bound}, and for which no number smaller than $e$ is an upper bound:

$$\Big\{ \hspace{5.0in} \Big\}$$
\vfill
\item The number $e$ is usually defined, for example during calculus II, as
	$$e = \sum_{n=0}^\infty \frac{1}{n!}$$
This gives another set of rational numbers which are less than $e$ but for which no number less than $e$ is also an upper bound.  That is, use the partial sums; you may use summation notation:

$$\Big\{ \hspace{5.0in} \Big\}$$
\vfill
\clearpage\newpage
\item In \textbf{4} and \textbf{5} it says ``set of rational numbers''.  Of course, they are also \emph{sequences} of rational numbers.  What is the difference?  Write a sentence or two.

\vspace{1.0in}
\item Write a sequence $(a_n)$ for \textbf{4}, and then a sequence $(b_n)$ for \textbf{5}.  For the latter you may use partial sum notation.
\vfill
\item Do the same as in \textbf{4}, but for the irrational square root of $2$, i.e.~$\sqrt{2} = 1.4142135623730950488\dots$.

$$\Big\{ \hspace{5.0in} \Big\}$$

\bigskip
\item Suppose you do not ``know'' the decimal expansion for $\sqrt{2}$.  Describe in words how you might construct a specific set $S$ of rational numbers for which $y=\sqrt{2}$ is an upper bound, and for which no smaller number is an upper bound.  (\emph{There are many correct answers.})
\vfill
\end{enumerate}

\end{document}
