\documentclass[12pt]{amsart}
\addtolength{\topmargin}{-0.6in} % usually -0.25in
\addtolength{\textheight}{1.1in} % usually 1.25in
\addtolength{\oddsidemargin}{-0.7in}
\addtolength{\evensidemargin}{-0.7in}
\addtolength{\textwidth}{1.5in} %\setlength{\parindent}{0pt}

\newcommand{\normalspacing}{\renewcommand{\baselinestretch}{1.04}\tiny\normalsize}

\normalspacing

% macros
\usepackage{amssymb,xspace}
\usepackage{tikz}
\usepackage[pdftex,colorlinks=true]{hyperref}


\newtheorem*{thm}{Theorem}
\newtheorem*{lem}{Lemma}

\newcommand{\mtt}{\texttt}
\newcommand{\mtl}[1]{{\texttt{>>#1}}}
\usepackage{alltt}
\usepackage{fancyvrb}

\newcommand{\bu}{\mathbf{u}}
\newcommand{\bv}{\mathbf{v}}

\newcommand{\CC}{{\mathbb{C}}}
\newcommand{\QQ}{{\mathbb{Q}}}
\newcommand{\RR}{{\mathbb{R}}}
\newcommand{\ZZ}{{\mathbb{Z}}}
\newcommand{\ZZn}{{\mathbb{Z}}_n}
\newcommand{\NN}{{\mathbb{N}}}

\newcommand{\eps}{\epsilon}
\newcommand{\grad}{\nabla}
\newcommand{\lam}{\lambda}
\newcommand{\ip}[2]{\mathrm{\left<#1,#2\right>}}
\newcommand{\erf}{\operatorname{erf}}

\renewcommand{\Re}{\operatorname{Re}}
\renewcommand{\Im}{\operatorname{Im}}
\newcommand{\Arg}{\operatorname{Arg}}

\newcommand{\Span}{\operatorname{span}}
\newcommand{\rank}{\operatorname{rank}}
\newcommand{\range}{\operatorname{range}}
\newcommand{\trace}{\operatorname{tr}}
\newcommand{\Null}{\operatorname{null}}

\newcommand{\Matlab}{\textsc{Matlab}\xspace}
\newcommand{\Octave}{\textsc{Octave}\xspace}
\newcommand{\pylab}{\textsc{pylab}\xspace}
\newcommand{\longMOP}{\textsc{Matlab}\big|\textsc{Octave}\big|\textsc{pylab}\xspace}
\newcommand{\MOP}{\textsc{M}\big|\textsc{O}\big|\textsc{p}\xspace}

\newcommand{\prob}[1]{\bigskip\noindent\large\textbf{#1.} \normalsize}
\newcommand{\bookprob}[1]{\bigskip\noindent\large\textbf{Exercise #1.} \normalsize}
\newcommand{\probpart}[1]{\smallskip\noindent\textbf{(#1)}\quad }
\newcommand{\aprobpart}[1]{\textbf{(#1)}\quad }


\newcommand*\circled[1]{\tikz[baseline=(char.base)]{
            \node[shape=circle,draw,inner sep=2pt] (char) {#1};}}


\begin{document}
\scriptsize \noindent Math 401 Introduction to Real Analysis \hfill Bueler: \emph{version 1}
\thispagestyle{empty}

\bigskip
\Large\textbf{\centerline{Final Exam: Prove 4 Theorems}}

\medskip
\large\textbf{\centerline{Tuesday, 9 December, 1:00pm--3:00pm, Chapman 204}}

\normalsize
\bigskip
The in-class Final Exam is different from the Midterms, but it is of modest length and it has a clear path for preparation.  You will prove 4 major theorems which were important during the semester.  The specific theorems are listed below, and you do have some choices.  I will grade your proofs for completeness, correctness, and clarity.

\smallskip
\textbf{You will have this document in your hand when you do the Final Exam.}  However, you must understand and remember what you want to write, because \textbf{you may NOT bring any notes or books or electronics to the Exam.}

\smallskip
To prepare for the Final Exam, you are strongly encouraged to draft and practice your proofs.  Please do these things during your preparation:
\begin{itemize}
\item read and understand the proofs in the textbook, and fill in any missing parts,
\item draft your proofs, 
\item get feedback on your drafts from other students/friends/family/pets, or me, and
\item decide in advance how you will remember enough details so that you can recreate the proofs during the Exam itself.
\end{itemize}

\smallskip
\noindent \hrulefill
\medskip

\noindent {\large \textbf{Directions.}} \, Prove theorems \textbf{A} and \textbf{B}.  Then choose one theorem from each category \textbf{C} and \textbf{D}, and prove it.  Put only one proof on each sheet of paper.  Write ``\textbf{A}'' or ``\textbf{C2}'' etc.~in the top left corner of the sheet, and write your name in the upper right.  Start each proof with ``Proof.'' and end it with ``\hspace{-4mm} $\qed$.''

\newcommand{\Thm}[3]{\bigskip\noindent \textbf{{\large #1.} \, #2} (Thm #3) \,}

\medskip
\Thm{A}{Monotone Convergence Theorem}{2.4.2}  If a sequence is monotone and bounded then it converges.

\Thm{B}{Bolzano-Weierstrass Theorem}{2.5.5}  Every bounded sequence contains a convergent subsequence.

\Thm{C1}{Density of $\QQ$}{1.4.3}  For every two real numbers $a$ and $b$ with $a<b$, there exists a rational $r\in\QQ$ satisfying $a<r<b$.

\Thm{C2}{$(0,1)$ is Uncountable}{1.6.1}  The open interval $(0,1) = \{x\in\RR\,:\,0<x<r\}$ is uncountable.

\Thm{D1}{Heine-Borel Theorem}{3.3.4}  A set $K\subseteq\RR$ is compact if and only if it is closed and bounded.

\Thm{D2}{Continuous Preserves Compactness}{4.4.1}  Let $f:A\to\RR$ be continuous on $A$.  If $K\subseteq A$ is compact then $f(K)$ is compact.

\Thm{D3}{Continuous Preserves Connectedness}{4.5.2}  Let $f:G\to\RR$ be continuous on $A$.  If $E\subseteq G$ is connected then $f(e)$ is connected.

\end{document}

