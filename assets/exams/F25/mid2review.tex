\documentclass[12pt]{amsart}
\addtolength{\topmargin}{-0.5in} % usually -0.25in
\addtolength{\textheight}{1.1in} % usually 1.25in
\addtolength{\oddsidemargin}{-0.7in}
\addtolength{\evensidemargin}{-0.7in}
\addtolength{\textwidth}{1.4in} %\setlength{\parindent}{0pt}

\newcommand{\normalspacing}{\renewcommand{\baselinestretch}{1.1}\tiny\normalsize}
\newcommand{\bigspacing}{\renewcommand{\baselinestretch}{1.15}\tiny\normalsize}
\newcommand{\tablespacing}{\renewcommand{\baselinestretch}{1.0}\tiny\normalsize}
\normalspacing

% packages
\usepackage{amssymb,xspace}
\usepackage[dvipsnames]{xcolor}
\usepackage[final]{graphicx}
\usepackage{fancyvrb}

\usepackage[pdftex,colorlinks=true,linkcolor=blue,urlcolor=blue]{hyperref}

\newtheorem*{thm}{Theorem}
\newtheorem*{lem}{Lemma}

% macros
\newcommand{\bu}{\mathbf{u}}
\newcommand{\bv}{\mathbf{v}}

\newcommand{\CC}{{\mathbb{C}}}
\newcommand{\NN}{{\mathbb{N}}}
\newcommand{\QQ}{{\mathbb{Q}}}
\newcommand{\RR}{{\mathbb{R}}}
\newcommand{\ZZ}{{\mathbb{Z}}}

\newcommand{\eps}{\epsilon}
\newcommand{\grad}{\nabla}
\newcommand{\lam}{\lambda}
\newcommand{\ip}[2]{\mathrm{\left<#1,#2\right>}}
\newcommand{\erf}{\operatorname{erf}}

\renewcommand{\Re}{\operatorname{Re}}
\renewcommand{\Im}{\operatorname{Im}}
\newcommand{\Arg}{\operatorname{Arg}}

\newcommand{\Span}{\operatorname{span}}
\newcommand{\rank}{\operatorname{rank}}
\newcommand{\range}{\operatorname{range}}
\newcommand{\trace}{\operatorname{tr}}
\newcommand{\Null}{\operatorname{null}}

\newcommand{\ds}{\displaystyle}
\newcommand{\prob}[1]{\bigskip\noindent\large\textbf{#1.} \normalsize}


\begin{document}
\scriptsize \noindent Math 401 Introduction to Real Analysis (Bueler) \hfill \today; {\color{BrickRed} version 2}
\thispagestyle{empty}

\bigskip
\Large\textbf{\centerline{Review Guide for In-Class Midterm Exam 2}}

\Large\textbf{\centerline{(which is on Friday, 7 November 2025)}}

\bigskip
\normalsize
\normalspacing

Midterm Exam 2 is \emph{closed book} and \emph{closed notes}.  (No technology is allowed.  Please bring nothing but a writing implement.)  Exam questions will be of these types: prove propositions, state definitions and axioms, state theorems, and give examples with certain properties.  You are expected to use reasonable and common notation; it is often wise to define your terms for clarity.

The Exam will cover sections 2.1, 2.3, 2.4, 2.5, 2.6, 2.7, and 3.2 of the textbook.\footnote{S.~Abbott, \emph{Understanding Analysis}, 2nd edition, Springer Press 2015}  However, earlier material will inevitably arise, with completeness of the real numbers, the definition of supremum/infimum, and convergence of sequences all certainly needed.

This Review Guide list \emph{specific material that might appear on the exam}.  Material which is \emph{significantly} different from what is listed below will \emph{not} appear.  The Exam will be built on topics that have appeared on homework and in lecture, and closely related things.  I will \emph{not} ask you to ``state theorem 2.4.2,'' or anything like that which would require remembering locations in the book, instead asking you to ``state the Monotone Convergence Theorem.''  However, numbers are listed below for ease of locating.

\emph{Strongly recommended}:  Get together with other students and work through this Review Guide.  Be honest with yourself about what you can easily prove, versus what you should think harder about and/or practice.  Talk it through and learn!

\bigskip

\bigspacing
\noindent \textbf{Definitions}.  Be able to state and use all of the definitions listed on the \href{https://bueler.github.io/real/assets/exams/F25/mid1review.pdf}{Review Guide for Midterm Exam 1}.  In addition, be able to state and use all of these definitions:

\begin{itemize}
\item $V_\eps(a) = (a-\eps,a+\eps)$ \quad $\eps$-neighborhood of $a$
\item a sequence is increasing, decreasing, or monotone
\item partial sum of an infinite series
\item convergence of an infinite series (\emph{is} convergence of the sequence of partial sums)
\item harmonic series $\ds \sum_{n=1}^\infty \frac{1}{n}$ and $p$-series $\ds \sum_{n=1}^\infty \frac{1}{n^p}$
\item arithmetic mean ($\frac{1}{2}(a+b)$) and geometric mean ($\sqrt{ab}$)
\item subsequence of a sequence
\item Cauchy sequence
\item geometric sequence
\item alternating series
\item absolutely-convergent and conditionally-convergent series
\item open set
\item arbitrary union and intersection of a collection of sets (see Problem 49, A7)
\item limit point of a set
\item closed set
\item isolated point of a set
\item closure of a set
\end{itemize}


\newcommand{\pne}{\hfill {\footnotesize $\leftarrow$ \textbf{proof not expected}}}

\bigskip
\noindent \textbf{Theorems}.  Be able to state, apply, and prove\footnote{Except if I say \textbf{proof not expected}, of course.  Even though I won't ask you to prove these during the in-class exam, please read and understand these proofs!} these theorems.
\begin{itemize}
\item algebraic limit theorem (Thm 2.3.3) \pne
\item order limit theorem (Thm 2.3.4) \pne
\item monotone convergence theorem (Thm 2.4.2)
\item $p$-series converge if and only if $p>1$ (Cor 2.4.7) \pne
\item subsequences of a convergent sequence converge (Thm 2.5.2)
\item Bolzano-Weierstrauss theorem (Thm 2.5.5) \pne
\item convergent sequences are Cauchy sequences (Thm 2.6.2)
\item Cauchy sequences are bounded (Lem 2.6.3)
\item Cauchy criterion (Thm 2.6.4)
\item (easy) algebraic limit theorem for series (Thm 2.7.1)
\item Cauchy criterion for series (Thm 2.7.2)
\item divergence test: if $\sum a_n$ converges then $a_n\to 0$ (Thm 2.7.3)
\item comparison test (Thm 2.7.4)
\item absolute convergence test (Thm 2.7.6)
\item alternating series test (Thm 2.7.7) \pne
\item absolute convergence allows rearrangement (Thm 2.7.10) \pne
\item arbitrary unions and finite intersections of open sets are open (Thm 3.2.3)
\item De Morgan's laws for arbitrary unions and intersections (see Problem 49, A7)
\item a limit point of $A$ is a limit of a sequence from $A$ (Thm 3.2.5)
\item a set is closed if and only if every Cauchy sequence converges (Thm 3.2.8)
\item every real number is the limit of a rational sequence (Thm 3.2.10)
\item the closure is a closed set (Thm 3.2.12)
\item a set $F$ is closed if and only if $F^c$ is open (Thm 3.2.13)
\item arbitrary intersections and finite unions of closed sets are closed (Thm 3.2.14)
\end{itemize}


\bigskip
\noindent \textbf{Examples}.  Every numbered Example in the identified textbook sections is fair game, as are closely-related examples.  Every question on homework Assignments 5--7 which mentions an example sequence or series or set, or is of the form ``provide an example (with these properties),'' is fair game, as are closely-related examples.

\vfill

\end{document}

