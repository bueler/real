\documentclass[12pt]{amsart}
\addtolength{\topmargin}{-0.5in} % usually -0.25in
\addtolength{\textheight}{1.1in} % usually 1.25in
\addtolength{\oddsidemargin}{-0.7in}
\addtolength{\evensidemargin}{-0.7in}
\addtolength{\textwidth}{1.4in} %\setlength{\parindent}{0pt}

\newcommand{\normalspacing}{\renewcommand{\baselinestretch}{1.1}\tiny\normalsize}
\newcommand{\bigspacing}{\renewcommand{\baselinestretch}{1.15}\tiny\normalsize}
\newcommand{\tablespacing}{\renewcommand{\baselinestretch}{1.0}\tiny\normalsize}
\normalspacing

% macros
\usepackage{amssymb,xspace}
\usepackage[pdftex,colorlinks=true,linkcolor=blue,urlcolor=blue]{hyperref}
\usepackage[dvipsnames]{xcolor}
\usepackage[final]{graphicx}
\newcommand{\regfigure}[3]{\includegraphics[height=#2in,width=#3in]{#1.eps}}

\newtheorem*{thm}{Theorem}
\newtheorem*{lem}{Lemma}

\usepackage{fancyvrb}

\newcommand{\bu}{\mathbf{u}}
\newcommand{\bv}{\mathbf{v}}

\newcommand{\CC}{{\mathbb{C}}}
\newcommand{\NN}{{\mathbb{N}}}
\newcommand{\QQ}{{\mathbb{Q}}}
\newcommand{\RR}{{\mathbb{R}}}
\newcommand{\ZZ}{{\mathbb{Z}}}

\newcommand{\eps}{\epsilon}
\newcommand{\grad}{\nabla}
\newcommand{\lam}{\lambda}
\newcommand{\ip}[2]{\mathrm{\left<#1,#2\right>}}
\newcommand{\erf}{\operatorname{erf}}

\renewcommand{\Re}{\operatorname{Re}}
\renewcommand{\Im}{\operatorname{Im}}
\newcommand{\Arg}{\operatorname{Arg}}

\newcommand{\Span}{\operatorname{span}}
\newcommand{\rank}{\operatorname{rank}}
\newcommand{\range}{\operatorname{range}}
\newcommand{\trace}{\operatorname{tr}}
\newcommand{\Null}{\operatorname{null}}

\newcommand{\prob}[1]{\bigskip\noindent\large\textbf{#1.} \normalsize}
\newcommand{\bookprob}[1]{\bigskip\noindent\large\textbf{Exercise #1.} \normalsize}
\newcommand{\probpart}[1]{\smallskip\noindent\textbf{(#1)}\quad }
\newcommand{\aprobpart}[1]{\textbf{(#1)}\quad }


\begin{document}
\scriptsize \noindent Math 401 Introduction to Real Analysis (Bueler) \hfill \today
\thispagestyle{empty}

\bigskip
\Large\textbf{\centerline{Review Guide for In-Class Midterm Exam 2,}}

\Large\textbf{\centerline{which is on Friday, 7 November 2025}}

\bigskip
\normalsize
\normalspacing

Midterm Exam 2 is \emph{closed book} and \emph{closed notes}.  (No technology is allowed.  Please bring nothing but a writing implement.)  Exam questions will be of these types: prove propositions, state definitions and axioms, state theorems, and give examples with certain properties.  You are expected to use reasonable and common notation; it is often wise to define your terms for clarity.

The Exam will cover sections 2.1, 2.3, 2.4, 2.5, 2.6, 2.7, 3.2, and 3.3 of the textbook.\footnote{S.~Abbott, \emph{Understanding Analysis}, 2nd edition, Springer Press 2015}  However, earlier material will inevitably arise, with completeness of the real numbers, the definition of supremum/infimum, and convergence of sequences all certainly needed.\footnote{Therefore, please also see the \href{https://bueler.github.io/real/assets/exams/F25/mid1review.pdf}{Review Guide for Midterm Exam 1}.}

This Review Guide list \emph{specific material that might appear on the exam}.  Material which is \emph{significantly} different from what is listed below will \emph{not} appear.  The Exam will be built on topics that have appeared on homework and in lecture, and closely related things.  I will \emph{not} ask you to ``state theorem 2.4.2,'' or anything like that which would require remembering locations in the book, instead asking you to ``state the Monotone Convergence Theorem.''  (However, numbers are listed below for easy of locating.)

\emph{Strongly recommended}:  Get together with other students and work through this Review Guide.  Be honest with yourself about what you can easily prove, versus what you should think harder about and/or practice.  Talk it through and learn!

\bigskip

\bigspacing
\noindent \textbf{Definitions}.  Be able to state and use the definition:

FIXME
\begin{itemize}
\item natural numbers $\NN$, integers $\ZZ$, rational numbers $\QQ$, irrational number
\item \emph{all definitions in section 1.2:} set union, set intersection, empty set, disjoint, set inclusion, set complement, function, domain of a function, range of a function, absolute value function, triangle inequality
\item one-to-one function, onto function, bijection
\item upper bound, lower bound, bounded above, bounded below (for real sets)
\item $\sup A$, $\inf A$ for a real set $A$
\item maximum, minimum for a real set
\item a subset is dense in $\RR$
\item finite set, infinite set
\item countable set, uncountable set
\item $A\sim B$: the sets $A,B$ have the same cardinality
\item power set of a set
\item algebraic real number (Exercise 1.5.9)
\item real sequence
\item a real sequence converges ($\lim_{n\to\infty} a_n = a$) or diverges
\item $\eps$-neighborhood of $a$
\item bounded sequence
\end{itemize}

\bigskip
\noindent \textbf{Concepts}.  

FIXME

Of course, you will be required to prove things, including one-way implications ($P\implies Q$) and equivalences ($P\iff Q$).  For the latter goal one generally proves each direction in turn, and you are expected to make it clear which direction is which.  Similarly, you might be asked to show set inclusion ($A \subseteq B$) or set equality ($A=B$), and for the latter one often shows inclusion both ways.

In addition, be able to use these proof techniques when appropriate:
\begin{itemize}
\item negate a proposition including ``for all'' and ``there exists'' quantifiers
\item proof by contradiction
\item proof by mathematical induction
\end{itemize}
and be able to state
\begin{itemize}
\item the axiom of completeness
\end{itemize}
\bigskip

\newcommand{\pne}{\hfill {\footnotesize $\leftarrow$ \textbf{proof not expected}}}
\noindent \textbf{Theorems}.  

FIXME

Be able to use and prove\footnote{Except if I say \textbf{proof not expected}, of course.  But please read and understand these proofs!} these theorems.
\begin{itemize}
\item there is no rational number whose square is 2 (Thm 1.1.1)
\item triangle inequality (Example 1.2.5)
\item two real numbers are equal if and only if \dots (Thm 1.2.6)
\item if $f:\RR\to\RR$ is a function and $A,B$ are real subsets then $f(A\cup B) = f(A) \cup f(B)$ and $f(A\cap B) \subseteq f(A) \cap f(B)$ (Exercise 1.2.7)
\item equivalent statement of $s=\sup A$ using $\eps>0$ (Lem 1.3.8)
\item nested interval property (Thm 1.4.1)
\item Archimedean properties (Thm 1.4.2)
\item density of $\QQ$ in $\RR$ (Thm 1.4.3)
\item there exists $\alpha\in\RR$ so that $\alpha^2=2$ (Thm 1.4.5) \pne
\item $\QQ$ is countable (Thm 1.5.6(i))
\item $\RR$ is uncountable, \emph{via nested interval} (Thm 1.5.6(ii)) \pne
\item if $A\subseteq B$ and $B$ is countable \dots (Thm 1.5.7)
\item countable union of finite sets is countable
\item finite union of countable sets is countable (Thm 1.5.8(i))
\item countable union of countable sets is countable (Thm 1.5.8ii) \pne
\item open interval $(0,1)$ is uncountable (Thm 1.6.1)
\item Cantor's theorem (Thm 1.6.2) \pne
\item limits are unique (Thm 2.2.7)
\item convergent sequences are bounded (Thm 2.3.2)
\item algebraic limit theorem (Thm 2.3.3)
\item order limit theorem (Thm 2.3.4)
\end{itemize}


\bigskip
\noindent \textbf{Examples}.  

FIXME

Every numbered Example, in the sections we covered, is fair game, as are closely-related examples.  Every question on the homework Assignments 1--4, of the form ``provide an example (with these properties),'' is fair game, as are closely-related examples.

\vfill

\end{document}

