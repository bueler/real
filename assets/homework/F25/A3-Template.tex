% <-- a percent symbol indicates a comment which does not affect the output of LaTeX
% you can leave the preamble alone, from here ...
\documentclass[12pt]{article}

\usepackage{amssymb,amsmath,amsthm}
\usepackage[top=1in, bottom=1in, left=1.25in, right=1.25in]{geometry}
\usepackage{enumerate,palatino}
\usepackage[final]{graphicx}
\usepackage[colorlinks=true,citecolor=blue,linkcolor=red,urlcolor=blue]{hyperref}

\newtheorem{problem}{Problem}
% ... to here

% shortcuts for blackboard bold number sets (reals, integers, etc.)
\newcommand{\QQ}{\ensuremath{\mathbb Q}}
\newcommand{\RR}{\ensuremath{\mathbb R}}
\newcommand{\NN}{\ensuremath{\mathbb N}}
\newcommand{\ZZ}{\ensuremath{\mathbb Z}}

% feel free to add more shortcuts here


\begin{document}
% replace with your name, but otherwise leave this header alone, from here ...
\small
\noindent \textsc{Math 401: Homework Assignment 3} \hfill YOUR NAME HERE

\normalsize
\bigskip
% ... to here

\setcounter{problem}{13}


\begin{problem} % Problem 14
Suppose $A,B$ are disjoint sets with $A\cup B = \RR$, and suppose that $a<b$ for all $a\in A$ and $b\in B$.  Then there exists $c\in\RR$ such that $x\le c$ for $x\in A$ and $x\ge c$ for $x\in B$.
\end{problem}

% PLEASE READ:  This is a cleaner version of Exercise 1.3.10 (a).  Prove the above claim using the Axiom of Completeness.

\begin{proof}
% FILL IN
\end{proof}


\begin{problem} % Problem 15
Here is example which shows that the claim in Problem 14 is false if $\RR$ is replaced, in both instances, by the set of rationals $\QQ$:
% PLEASE READ:  This is a cleaner version of Exercise 1.3.10 (c).
% FILL IN YOUR EXAMPLE
\end{problem}


\begin{problem} % Problem 16
Let $a<b$ be real numbers.  Define the set $T=\QQ \cap [a,b]$.  Then $\sup T = b$.
\end{problem}

% PLEASE READ:  This is Exercise 1.4.4.

\begin{proof}
% FILL IN
\end{proof}


\begin{problem} % Problem 14
    $$\bigcap_{n=1}^\infty (0,1/n)=\emptyset.$$
\end{problem}

% PLEASE READ:  

\begin{proof}
% FILL IN
\end{proof}


\begin{problem} % Problem 14
    $$\bigcap_{n=1}^\infty (0,1/n)=\emptyset.$$
\end{problem}

% PLEASE READ:  

\begin{proof}
% FILL IN
\end{proof}


\begin{problem} % Problem 14
    $$\bigcap_{n=1}^\infty (0,1/n)=\emptyset.$$
\end{problem}

% PLEASE READ:  

\begin{proof}
% FILL IN
\end{proof}


\begin{problem} % Problem 14
    $$\bigcap_{n=1}^\infty (0,1/n)=\emptyset.$$
\end{problem}

% PLEASE READ:  

\begin{proof}
% FILL IN
\end{proof}


\end{document}