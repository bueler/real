% <-- a percent symbol indicates a comment which does not affect the output of LaTeX
% you can leave the preamble alone, from here ...
\documentclass[12pt]{article}

\usepackage{amssymb,amsmath,amsthm}
\usepackage[top=1in, bottom=1in, left=1.25in, right=1.25in]{geometry}
\usepackage{enumerate,palatino}
\usepackage[final]{graphicx}
\usepackage[colorlinks=true,citecolor=blue,linkcolor=red,urlcolor=blue]{hyperref}

\newtheorem{problem}{Problem}
% ... to here

% shortcuts for blackboard bold number sets (reals, integers, etc.)
\newcommand{\II}{\ensuremath{\mathbb I}}
\newcommand{\NN}{\ensuremath{\mathbb N}}
\newcommand{\QQ}{\ensuremath{\mathbb Q}}
\newcommand{\RR}{\ensuremath{\mathbb R}}
\newcommand{\ZZ}{\ensuremath{\mathbb Z}}

% feel free to add more shortcuts here


\begin{document}
% replace with your name, but otherwise leave this header alone, from here ...
\small
\noindent \textsc{Math 401: Homework Assignment 4} \hfill YOUR NAME HERE

\normalsize
\bigskip
% ... to here

\setcounter{problem}{21}


\begin{problem} % Problem 22
If $f:A\to B$ has an inverse function then $f$ is onto and $f$ is one-to-one.
\end{problem}

% PLEASE READ:  Recall that g:B->A is an inverse function of f:A->B if g(f(a))=a for all a in A and if f(g(b))=b for all b in B.  You will want to write separate paragraphs in your proof for the two results, something like "First, to prove $f$ is onto we ..." and then "To prove $f$ is one-to-one ..."

\begin{proof}
% FILL IN
\end{proof}


\begin{problem} % Problem 23
xx
\end{problem}

% PLEASE READ:  yy

\begin{proof}
% FILL IN
\end{proof}


\begin{problem} % Problem 24
xx
\end{problem}

% PLEASE READ:  yy

\begin{proof}
% FILL IN
\end{proof}


\begin{problem} % Problem 25
xx
\end{problem}

% PLEASE READ:  yy

\begin{proof}
% FILL IN
\end{proof}


\begin{problem} % Problem 26
xx
\end{problem}

% PLEASE READ:  yy

\begin{proof}
% FILL IN
\end{proof}


\begin{problem} % Problem 27
xx
\end{problem}

% PLEASE READ:  yy

\begin{proof}
% FILL IN
\end{proof}


\end{document}