% <-- a percent symbol indicates a comment which does not affect the output of LaTeX
% you can leave the preamble alone, from here ...
\documentclass[12pt]{article}

\usepackage{amssymb,amsmath,amsthm}
\usepackage[top=1in, bottom=1in, left=1.25in, right=1.25in]{geometry}
\usepackage{enumerate,palatino}
\usepackage[final]{graphicx}
\usepackage[colorlinks=true,citecolor=blue,linkcolor=red,urlcolor=blue]{hyperref}

\newtheorem{problem}{Problem}
% ... to here

% shortcuts for blackboard bold number sets (reals, integers, etc.)
\newcommand{\QQ}{\ensuremath{\mathbb Q}}
\newcommand{\RR}{\ensuremath{\mathbb R}}
\newcommand{\NN}{\ensuremath{\mathbb N}}
\newcommand{\ZZ}{\ensuremath{\mathbb Z}}

% feel free to add more shortcuts around here

\begin{document}
% replace with your name, but otherwise leave this header alone, from here ...
\small
\noindent \textsc{Math 401: Homework Assignment 2} \hfill YOUR NAME HERE

\normalsize
\bigskip
% ... to here

\setcounter{problem}{5}



\begin{problem} % Problem 6
    $$\bigcap_{n=1}^\infty (0,1/n)=\emptyset.$$
\end{problem}

% PLEASE READ:  Let $S = \cap_{n=1}^\infty (0,1/n)$.  Show by contradiction that $x\in S$ is false for every $x\in\RR$?

\begin{proof}
% FILL IN
\end{proof}


\begin{problem} % Problem 7
Given a function $f$ and a subset $A$ of its domain, consider the image $f(A) = \{f(x) : x \in A\}$.

\begin{itemize}
\item[(a)] An example of two sets $A$ and $B$, and a function $f$, for which $f(A \cap B) \neq f(A) \cap f(B)$ is

% PUT EXAMPLE HERE

% IN PART (b), FORM A CONJECTURE, AND THEN PROVE AS A PROPOSITION, A RELATIONSHIP BETWEEN  f(A \cup B) and f(A) \cup f(B)
\item[(b)] If $A$, $B$ are subsets of the domain of $f$ then

% COMPLETE STATEMENT OF PROPOSITION

\begin{proof}
% FILL IN
\end{proof}
\end{itemize}
\end{problem}


\begin{problem} % Problem 8
If $a\in\RR$ is an upper bound for $A\subset \RR$, and if $a$ is also an element of $A$, then $a = \sup A$.
\end{problem}

\begin{proof}
% FILL IN
\end{proof}


\begin{problem} % Problem 9
\begin{enumerate}[(a)]
\item Let $A=\{m/n: \text{ $m,n\in\NN$ with $m<n$} \}$.  Then $\inf A = $ and $\sup A = $.
\item Let $B=\{(-1)^m/n: n,m\in\NN\}$.  Then $\inf B = $ and $\sup B = $.
\item Let $C=\{n/(3n+1): n\in\NN\}$.  Then $\inf C = $ and $\sup C = $.
\item Let $D=\{m/(m+n):m,n\in\NN\}$.  Then $\inf D = $ and $\sup D = $.
\end{enumerate}
\end{problem}

% PLEASE READ:  FILL IN THE INFIMA AND SUPREMA ABOVE.  YOU DO NOT HAVE TO PROVE ANYTHING.



\end{document}