% <-- a percent symbol indicates a comment which does not affect the output of LaTeX
% you can leave the preamble alone, from here ...
\documentclass[12pt]{article}

\usepackage{amssymb,amsmath,amsthm}
\usepackage[top=1in, bottom=1in, left=1.25in, right=1.25in]{geometry}
\usepackage{enumerate,palatino}
\usepackage[final]{graphicx}
\usepackage[colorlinks=true,citecolor=blue,linkcolor=red,urlcolor=blue]{hyperref}

\newtheorem{problem}{Problem}
% ... to here

% shortcuts for blackboard bold number sets (reals, integers, etc.)
\newcommand{\II}{\ensuremath{\mathbb I}}
\newcommand{\NN}{\ensuremath{\mathbb N}}
\newcommand{\QQ}{\ensuremath{\mathbb Q}}
\newcommand{\RR}{\ensuremath{\mathbb R}}
\newcommand{\ZZ}{\ensuremath{\mathbb Z}}

\newcommand{\eps}{\ensuremath{\epsilon}}

% feel free to add more shortcuts here


\begin{document}
% replace with your name, but otherwise leave this header alone, from here ...
\small
\noindent \textsc{Math 401: Homework Assignment 5} \hfill YOUR NAME HERE

\normalsize
\bigskip
% ... to here

\setcounter{problem}{28}

\begin{problem} % Problem 29
Suppose $(x_n)_{n=1}^\infty$ converges.  Let $k \in \NN$.  The new sequence $(x_{n+k})_{n=1}^\infty$ also converges, and to the same limit.
\end{problem}

% PLEASE READ:  You will have to prove this directly from the definition of the convergence of sequences.

\begin{proof}
% FILL IN
\end{proof}


\begin{problem} % Problem 30
Give an example of each of the following, or state that such a request is impossible.  In the latter case, identify specific theorem(s) that justify your statement.

\renewcommand{\labelenumi}{(\alph{enumi})}
\begin{enumerate}
\item sequences $(x_n)$ and $(y_n)$, which both diverge, where the sum $(x_n+y_n)$ converges

% FILL IN

\item a convergent sequence $(x_n)$, and a divergent sequence $(y_n)$, where $(x_n+y_n)$ converges

% FILL IN

\item a convergent sequence $(b_n)$, with $b_n\ne 0$ for all $n$, such that $(1/b_n)$ diverges

% FILL IN

\item sequences $(x_n)$ and $(y_n)$, where $(x_n y_n)$ and $(x_n)$ converge but $(y_n)$ does not

% FILL IN

\end{enumerate}
\end{problem}


\begin{problem} % Problem 31
If $a\ge 0$ and $b\ge 0$ then \, $\displaystyle \sqrt{ab} \le \frac{1}{2}\left(a+b\right)$.
\end{problem}

% PLEASE READ:  This fact is called the arithmetic-geometric mean inequality.  The left side is the geometric mean (why?) and the right side is the usual mean (i.e. average).  As a hint on the proof, square the sides of what you want to prove, and play around.  The logical starting point of your proof will be the true fact that the square of a certain real number is nonnegative.

\begin{proof}
% FILL IN
\end{proof}


\begin{problem} % Problem 32
Consider the real sequence generated by setting $x_1=2$ and then
	$$x_{n+1} = \frac{1}{2}\left(x_n + \frac{2}{x_n}\right).$$

% PLEASE READ:  In (a) you may use the arithmetic-geometric mean inequality from Problem 31.  In (b) you will want to use (a) and a major theorem in section 2.4.  That is, be careful to establish that the limit exists before showing it is $\sqrt{2}$.  A proof by induction of monotocity is appropriate.  You will also use Problem 29.
%
% What you are proving here could be phrased as "Newton's method applied to solve x^2-2=0, starting with the guess x_1=2, converges to the root \sqrt{2}".  (Explain this, on your own time, from the general Newton's method formula.)  In fact, the convergence is stunningly fast, as you can see (on your own time) by using a calculator to get the first 6 terms.  However, your proof if (b) says nothing about *speed* of convergence.

\renewcommand{\labelenumi}{(\alph{enumi})}
\begin{enumerate}
\item The sequence $(x_n)$ is bounded below by $\sqrt{2}$.

\begin{proof}
% FILL IN
\end{proof}

\item $\lim_{n\to\infty} x_n = \sqrt{2}$.

\begin{proof}
% FILL IN
\end{proof}

\end{enumerate}
\end{problem}


\begin{problem} % Problem 33
The sequence $\displaystyle \sqrt{2}, \sqrt{2+\sqrt{2}}, \sqrt{2 + \sqrt{2+\sqrt{2}}}, \dots$ converges to $X$.
\end{problem}

% PLEASE READ:  I will do a problem like this in lecture.  Step 1 is to write down a recurrence: x_1=\sqrt{2} and x_{n+1}=f(x_n).  That is, establish how each term is computed from the last and thereby find f(x).  Then show that the sequence is increasing and bounded.  Then establish that the limit exists.  Then find its limit $X$; of course, fill in a particular value for $X$.  Both in Problem 32 and this problem, it is just fine to start by computing some terms in the sequence, but that does not prove anything.

\begin{proof}
% FILL IN
\end{proof}


\begin{problem} % Problem 34
For each series, find an explicit formula for the partial sums, and determine if the series converges.

\renewcommand{\labelenumi}{(\alph{enumi})}
\begin{enumerate}
\item $\displaystyle \sum_{n=1}^\infty \frac{1}{2^n}$

% FILL IN

\item $\displaystyle \sum_{n=1}^\infty \frac{1}{n(n+1)}$

% FILL IN

\item $\displaystyle \sum_{n=1}^\infty \log\left(\frac{n+1}{n}\right)$  %  Note log(x)=ln(x) is the natural logarithm.

% FILL IN

\end{enumerate}
\end{problem}


\begin{problem} % Problem 35
\phantom{foo}

% PLEASE READ:  In (a) you should, of course, consider the sequences of partial sums.  The idea of monotone sequences will arise.  In (b), use the known fact that the harmonic series diverges (which you do not need to prove) and part (a) to conclude.  Note that part (a) is a comparison test for series.

\renewcommand{\labelenumi}{(\alph{enumi})}
\begin{enumerate}
\item Suppose $0 \le a_n \le b_n$.  If \, $\displaystyle \sum_{n=1}^\infty a_n$ diverges then $\displaystyle \sum_{n=1}^\infty b_n$ diverges.

\begin{proof}
% FILL IN
\end{proof}

\item $\displaystyle \sum_{n=1}^\infty \frac{1}{\sqrt{n}}$ diverges.

\begin{proof}
% FILL IN
\end{proof}

\end{enumerate}
\end{problem}


\begin{problem} % Problem 36
Give an example of each of the following, or argue that it is impossible.

\renewcommand{\labelenumi}{(\alph{enumi})}
\begin{enumerate}
\item A sequence that has a subsequence that is bounded, but which contains no subsequence which converges.

% FILL IN

\item A sequence that does not contain $0$ or $1$ as a term, but which contains subsequences which converge to each of these values.

% FILL IN

\item A sequence that contains subsequences converging to every point in the infinite set $\{1,1/2,1/3,1/4,\dots\}$.

% FILL IN

\end{enumerate}
\end{problem}


\begin{problem} % Problem 37
Let $(a_n)$ be a bounded sequence.  Define the set
	$$S = \left\{x\in\RR\,:\, x < a_n \,\text{ for infinitely many terms } a_n\right\}.$$
Then $S$ is bounded above, and there exists a subsequence $(a_{n_k})$ which converges to $\sup S$.
\end{problem}

\begin{proof}
% FILL IN
\end{proof}

\end{document}