% <-- a percent symbol indicates a comment which does not affect the output of LaTeX
% you can leave the preamble alone, from here ...
\documentclass[12pt]{article}

\usepackage{amssymb,amsmath,amsthm}
\usepackage[top=1in, bottom=1in, left=1.25in, right=1.25in]{geometry}
\usepackage{enumerate,palatino}
\usepackage[final]{graphicx}
\usepackage[colorlinks=true,citecolor=blue,linkcolor=red,urlcolor=blue]{hyperref}

\newtheorem{problem}{Problem}
% ... to here

% shortcuts for blackboard bold number sets (reals, integers, etc.)
\newcommand{\II}{\ensuremath{\mathbb I}}
\newcommand{\NN}{\ensuremath{\mathbb N}}
\newcommand{\QQ}{\ensuremath{\mathbb Q}}
\newcommand{\RR}{\ensuremath{\mathbb R}}
\newcommand{\ZZ}{\ensuremath{\mathbb Z}}

\newcommand{\eps}{\ensuremath{\epsilon}}

% feel free to add more shortcuts here


\begin{document}
% replace with your name, but otherwise leave this header alone, from here ...
\small
\noindent \textsc{Math 401: Homework Assignment 5} \hfill YOUR NAME HERE

\normalsize
\bigskip
% ... to here

\setcounter{problem}{28}

%plan: easy comparison test with appl to sum 1/sqrt{n}, exer 2.5.1, exer 2.5.9

\begin{problem} % Problem 29
Suppose $(x_n)_{n=1}^\infty$ converges.  Then, for any fixed $k=1,2,\dots$, the new sequence $(x_{n+k})_{n=1}^\infty$ also converges, and to the same limit.
\end{problem}

% PLEASE READ:  You will have to prove this directly from the definition of the convergence of sequences.

\begin{proof}
% FILL IN
\end{proof}


\begin{problem} % Problem 30
If $a\ge 0$ and $b\ge 0$ then \, $\displaystyle \sqrt{ab} \le \frac{1}{2}\left(a+b\right)$.
\end{problem}

% PLEASE READ:  This fact is called the arithmetic-geometric mean inequality.  The left side is the geometric mean (why?) and the right side is the usual mean (i.e. average).  As a hint on the proof, square the sides of what you want to prove, and play around.  The logical starting point of your proof will be the true fact that the square of a certain real number is nonnegative.

\begin{proof}
% FILL IN
\end{proof}


\begin{problem} % Problem 31
Consider the real sequence generated by setting $x_1=2$ and then
	$$x_{n+1} = \frac{1}{2}\left(x_n + \frac{2}{x_n}\right).$$

% PLEASE READ:  In (a) you may use the arithmetic-geometric mean inequality from Problem 30.  In (b) you will want to use (a) and a major theorem in section 2.4; be careful to establish that the limit exists before showing it is $\sqrt{2}$.  You will also use Problem 29.  What you are proving here could be phrased as "Newton's method applied to solve x^2-2=0, starting with the guess x_1=2, converges to the root \sqrt{2}."

\renewcommand{\labelenumi}{(\alph{enumi})}
\begin{enumerate}
\item The sequence $(x_n)$ is bounded below by $\sqrt{2}$.

\begin{proof}
% FILL IN
\end{proof}

\item $\lim_{n\to\infty} x_n = \sqrt{2}$.

\begin{proof}
% FILL IN
\end{proof}

\end{enumerate}
\end{problem}


\begin{problem} % Problem 32
Let $x_1=3$ and
	$$x_{n+1} = \frac{1}{4-x_1}.$$
Then $(x_n)$ converges to $X$.
\end{problem}

% PLEASE READ:  Be careful to establish that the limit exists before giving its value $X$, and confirming that; of course, fill in a particular value for $X$.  Some of the techniques from Problem 31 will also get used here, but not the arithmetic-geometric mean inequality.  Both in Problem 31 and this problem, it is just fine to start by computing some terms in the sequence, but that does not prove anything.

\begin{proof}
% FILL IN
\end{proof}


\begin{problem} % Problem 33
Give an example of each of the following, or state that such a request is impossible.  In the latter case, identify specific theorem(s) that justify your statement.

\renewcommand{\labelenumi}{(\alph{enumi})}
\begin{enumerate}
\item sequences $(x_n)$ and $(y_n)$, which both diverge, where the sum $(x_n+y_n)$ converges

% FILL IN

\item a convergent sequence $(x_n)$, and a divergent sequence $(y_n)$, where $(x_n+y_n)$ converges

% FILL IN

\item a convergent sequence $(b_n)$, with $b_n\ne 0$ for all $n$, such that $(1/b_n)$ diverges

% FILL IN

\item sequences $(x_n)$ and $(y_n)$, where $(x_n y_n)$ and $(x_n)$ converge but $(y_n)$ does not

% FILL IN

\end{enumerate}
\end{problem}


\begin{problem} % Problem 34
For each series, find an explicit formula for the partial sums, and determine if the series converges.

\renewcommand{\labelenumi}{(\alph{enumi})}
\begin{enumerate}
\item $\displaystyle \sum_{n=1}^\infty \frac{1}{2^n}$

% FILL IN

\item $\displaystyle \sum_{n=1}^\infty \frac{1}{n(n+1)}$

% FILL IN

\item $\displaystyle \sum_{n=1}^\infty \log\left(\frac{n+1}{n}\right)$

% PLEASE READ:  Note log(x)=ln(x) is the natural logarithm.
% FILL IN

\end{enumerate}
\end{problem}

\end{document}