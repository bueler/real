% <-- a percent symbol indicates a comment which does not affect the output of LaTeX
% you can leave the preamble alone, from here ...
\documentclass[12pt]{article}

\usepackage{amssymb,amsmath,amsthm}
\usepackage[top=1in, bottom=1in, left=1.25in, right=1.25in]{geometry}
\usepackage{enumitem,palatino}
\usepackage[final]{graphicx}
\usepackage[colorlinks=true,citecolor=blue,linkcolor=red,urlcolor=blue]{hyperref}

\newtheorem{problem}{Problem}
% ... to here

% shortcuts for blackboard bold number sets (reals, integers, etc.)
\newcommand{\II}{\ensuremath{\mathbb I}}
\newcommand{\NN}{\ensuremath{\mathbb N}}
\newcommand{\QQ}{\ensuremath{\mathbb Q}}
\newcommand{\RR}{\ensuremath{\mathbb R}}
\newcommand{\ZZ}{\ensuremath{\mathbb Z}}

\newcommand{\eps}{\ensuremath{\epsilon}}
\newcommand{\ds}{\displaystyle}

% feel free to add more shortcuts here


\begin{document}
% replace with your name, but otherwise leave this header alone, from here ...
\small
\noindent \textsc{Math 401: Homework Assignment 9} {\footnotesize [version 2]} \hfill YOUR NAME HERE

\normalsize
\bigskip
% ... to here

\setcounter{problem}{59}

\begin{problem} % Problem 60
Let $f,g,h$ satisfy $f(x)\le g(x)\le h(x)$ for all $x$ in some common domain $A$.  Assume $c$ is a limit point of $A$.  If $\lim_{x\to c} f(x)=L$ and $\lim_{x\to c} h(x)=L$ then $\lim_{x\to c} g(x)=L$.
\end{problem}

% PLEASE READ: This is a squeeze theorem, and it is Exercise 4.2.11 in the book.  Compare Exercise 2.3.3.

\begin{proof}
% FILL IN
\end{proof}


\begin{problem} % Problem 61
If $h:\RR \to \RR$ is a continuous function then the set $K=\{x\in \RR\,:\,h(x)=0\}$ is closed.
\end{problem}

% PLEASE READ: As expected, prove that K contains its limit points.  (Or prove that K^c is open.)

\begin{proof}
% FILL IN
\end{proof}


\begin{problem} % Problem 62
If $c$ is an isolated point of $A\subset\RR$, and if $f:A\to\RR$ is a function, then $f$ is continuous at $c$.
\end{problem}

% PLEASE READ: Read Definition 4.3.1 carefully.

\begin{proof}
% FILL IN
\end{proof}


\begin{problem} % Problem 63
The function $g:\RR\to\RR$ defined by $g(x) = \sqrt[3]{x}$ is continuous.
\end{problem}

% PLEASE READ: You will want to prove that g is continuous at c = 0, and separately prove it for every other point c \ne 0.  For the latter, the identity a^3 - b^3 = (a - b) (a^2 + ab + b^2) will be useful.

\begin{proof}
% FILL IN
\end{proof}


\begin{problem} % Problem 64
Dirichlet's function from Section 4.1, namely
	$$g(x) = \begin{cases} 1 &\text{ if } x \in \QQ, \\
	                       0 &\text{ if } x \notin \QQ \end{cases}$$
is not continuous at any $c\in\RR$.
\end{problem}

% PLEASE READ: On page 123 there is a suggestion about a good way to show a function is not continuous; feel free to use that technique.

\begin{proof}
% FILL IN
\end{proof}


\begin{problem} % Problem 65
The function
	$$h(x) = \begin{cases} 0 &\text{ if } x = 0, \\
	                       \sqrt{|x|} \cos(1/x) &\text{ otherwise}, \end{cases}$$
shown in the figure below, is continuous at zero.
\end{problem}

% PLEASE READ: Please sketch the graph of h(x), or generate the graph using a computer, and include that figure in your solution.  Consider making your figure a .jpg and then using this LaTeX:
%\begin{center}
%  \includegraphics[width=0.5\textwidth]{figure.jpg}
%\end{center}

\begin{proof}
% FILL IN
\end{proof}


\begin{problem} % Problem 66
Thomae's function from Section 4.1, namely
	$$t(x) = \begin{cases} 1   &\text{ if } x = 0, \\
	                       1/n &\text{ if } x \in \QQ \setminus \{0\} \text{ and } x=\pm m/n \text{ in lowest terms, with } n>0, \\
	                       0   &\text{ if } x \notin \QQ \end{cases}$$
is not continuous at any rational point $c\in\QQ$.
\end{problem}

% PLEASE READ: If you need the fact that every rational is the limit of an irrational sequence, please *justify* (i.e. prove) this fact.

\begin{proof}
% FILL IN
\end{proof}

\begin{problem} % Problem 67
Suppose $f:A \to \RR$ is continuous at $c\in A$.  Suppose that $g:B\to\RR$ has a domain satisfying $f(A) \subset B$, and that $g$ is continuous at $f(c)$.  Let
	$$h(x) = (g \circ f)(x) = g(f(x))$$
be the composition of functions. Then $h$ is continuous at $c$.
\end{problem}

% PLEASE READ: This is Theorem 4.3.9 in the book.  You can prove it either using the \epsilon-\delta definition of continuity, or the sequential characterization (Theorem 4.3.2 (iii)).

\begin{proof}
% FILL IN
\end{proof}

\end{document}