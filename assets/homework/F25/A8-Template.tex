% <-- a percent symbol indicates a comment which does not affect the output of LaTeX
% you can leave the preamble alone, from here ...
\documentclass[12pt]{article}

\usepackage{amssymb,amsmath,amsthm}
\usepackage[top=1in, bottom=1in, left=1.25in, right=1.25in]{geometry}
\usepackage{enumitem,palatino}
\usepackage[final]{graphicx}
\usepackage[colorlinks=true,citecolor=blue,linkcolor=red,urlcolor=blue]{hyperref}

\newtheorem{problem}{Problem}
% ... to here

% shortcuts for blackboard bold number sets (reals, integers, etc.)
\newcommand{\II}{\ensuremath{\mathbb I}}
\newcommand{\NN}{\ensuremath{\mathbb N}}
\newcommand{\QQ}{\ensuremath{\mathbb Q}}
\newcommand{\RR}{\ensuremath{\mathbb R}}
\newcommand{\ZZ}{\ensuremath{\mathbb Z}}

\newcommand{\eps}{\ensuremath{\epsilon}}
\newcommand{\ds}{\displaystyle}

% feel free to add more shortcuts here


\begin{document}
% replace with your name, but otherwise leave this header alone, from here ...
\small
\noindent \textsc{Math 401: Homework Assignment 8} \hfill YOUR NAME HERE

\normalsize
\bigskip
% ... to here

\setcounter{problem}{50}


\begin{problem} % Problem 51
If $K\subset\RR$ is compact and nonempty, then $\sup K$ and $\inf K$ both exist and are elements of $K$.
\end{problem}

\begin{proof}
% FILL IN
\end{proof}


\begin{problem} % Problem 52
What of the following sets are compact?  For those that are compact, give a brief justification.  For those that are not, show how the book's definition of compact (Definition 3.3.1) breaks down.  That is, give an example sequence in the set that does not contain a subsequence which converges to a point in the set.
\end{problem}

% PLEASE READ: Please read and think about Section 3.3.

\renewcommand{\labelenumi}{(\alph{enumi})}
\begin{enumerate}
\item $\NN$

% FILL IN
\item $\QQ\cap [0,1]$

% FILL IN
\item The Cantor set $C$

% FILL IN
\item $\ds \left\{1+\frac{1}{2^2}+\frac{1}{3^2}+\dots+\frac{1}{n^2}\,:\,n\in\NN\right\}$

% FILL IN
\item $\ds \left\{1,\frac{1}{2},\frac{2}{3},\frac{4}{5},\dots\right\}$

% FILL IN
\end{enumerate}


\begin{problem} % Problem 53
If a set $K \subset \RR$ is closed and bounded, then it is compact.
\end{problem}

% PLEASE READ: This asks you to prove the part of Theorem 3.3.4 that is not shown in the textbook.

\begin{proof}
% FILL IN
\end{proof}


\begin{problem} % Problem 54
Decide whether the following propositions are true or false.  If the claim is valid, supply a short proof.  If the claim is false, provide a counterexample.
\end{problem}

% PLEASE READ: Please read and think about Section 3.3.  Each time you introduce a proof, use a LaTeX proof environment (\begin{proof} ... \end{proof}).

\renewcommand{\labelenumi}{(\alph{enumi})}
\begin{enumerate}
\item The arbitrary union of compact sets is compact.

% FILL IN
\item The arbitrary intersection of compact sets is compact.

% FILL IN
\item Let $A$ be arbitrary, and let $K$ be compact.  Then $A\cap K$ is compact.

% FILL IN
\item If $F_1 \supseteq F_2 \supseteq F_3 \supseteq \dots$ is a nested sequence of nonempty closed sets then the intersection $\ds F = \bigcap_{n=1}^\infty F_n$ is nonempty.

% FILL IN
\end{enumerate}


\begin{problem} % Problem 55
Let $A$ and $B$ be nonempty subsets of $\RR$.  If there exist disjoint open sets $U$ and $V$ with $A\subseteq U$ and $B\subseteq V$, then $A$ and $B$ are separated.
\end{problem}

% PLEASE READ: 

\begin{proof}
% FILL IN
\end{proof}


\begin{problem} % Problem 56
A set $E\subset \RR$ is \emph{totally disconnected} if, given any two distinct points $x,y\in E$, there exist separated sets $A$ and $B$ with $x\in A$ and $y\in B$ and $E=A\cup B$.

The Cantor set $C$ is totally disconnected.
\end{problem}

% PLEASE READ: Prove that the Cantor set $C$ is totally disconnected by the following steps:
% 1. In section 3.1, $C$ is defined as an intersection of closed sets $C_n$, namely $C = \bigcap_{n=0}^\infty C_n$.  As the first steps of your proof, recall how $C_n$ are constructed as a finite union of disjoint closed intervals, and recall this intersection fact.
% 2. Given $x,y \in C$, with $x<y$, set $\epsilon=y-x$.  Explain why there must exist an $N$ large enough so that it is impossible for $x$ and $y$ to both belong to the same closed interval within $C_N$.
% 3. Now argue that $C$ is totally disconnected.

\begin{proof}
% FILL IN
\end{proof}


\begin{problem} % Problem 57
For each stated limit, find the largest possible $\delta$-neighborhood that is a proper response to the given $\eps$ challenge.  Note that $[[x]]$ denotes the greatest integer which is less than or equal to $x$.
\end{problem}

% PLEASE READ: The language of "\epsilon challenge" is briefly explained at the beginning of Section 4.2, but it should be clear from the meaning of the functional limit.

\renewcommand{\labelenumi}{(\alph{enumi})}
\begin{enumerate}
\item $\ds \lim_{x\to 3} 5x-6 = 9$, where $\eps=1$

% FILL IN
\item $\ds \lim_{x\to 4} \sqrt{x} = 2$, where $\eps=0.5$

% FILL IN
\item $\ds \lim_{x\to \pi} [[x]] = 3$, where $\eps=0.5$

% FILL IN
\end{enumerate}


\begin{problem} % Problem 58
Use the definition of functional limit in the textbook (Definition 4.2.1) to prove the following limit statements.
\end{problem}

\renewcommand{\labelenumi}{(\alph{enumi})}
\begin{enumerate}
\item $\ds \lim_{x\to 2} 3x+4 = 10$

\begin{proof}
% FILL IN
\end{proof}

\item $\ds \lim_{x\to 2} x^2 + x - 1 = 5$

\begin{proof}
% FILL IN
\end{proof}

\item $\ds \lim_{x\to 3} \frac{1}{x} = \frac{1}{3}$

\begin{proof}
% FILL IN
\end{proof}

\item $\ds \lim_{x\to 3} \frac{x^2-9}{x-3} = 6$

% PLEASE READ:  Justify any algebra which is essential to proving the limit.  What is the domain of the function f(x)=(x^2-9)/(x-3)?

\begin{proof}
% FILL IN
\end{proof}

\end{enumerate}


\begin{problem} % Problem 59
Let $g:A\to\RR$ and assume that $f$ is a bounded function on $A$.  Assume $c$ is a limit point of $A$.  If $\ds \lim_{x\to c} g(x)=0$ then $\ds \lim_{x\to c} f(x) g(x)=0$.
\end{problem}

\begin{proof}
% FILL IN
\end{proof}


\end{document}